\documentclass[12pt,
openright, 
oneside,
%twoside, %TCC: Se seu texto tem mais de 100 páginas, descomente esta linha e comente a anterior
a4paper,
brazil]{facom-ufu-abntex2}

%\usepackage[disable]{todonotes}
\usepackage[colorinlistoftodos]{todonotes}	%use a linha anterior para esconder os todos.
\usepackage{listings}
\usepackage{forest}
\usepackage{listings}

\newcommand{\codigo}{
\lstset{
	language=C++,
    basicstyle=\ttfamily\tiny,
    stringstyle=\color{red},
    commentstyle=\color{green},
    morecomment=[s][\color{blue}]{/**}{*/},
    extendedchars=true,
    showspaces=false,
    showstringspaces=false,
    numbers=left,
    numberstyle=\tiny,
    breaklines=true,
    breakautoindent=true,
    breakatwhitespace=true,
}}

\renewcommand{\lstlistingname}{Listagem}

\autor{José Augusto Bolina Lucas} %TCC
\data{2018}
\orientador{Lásaro Jonas Camargos} %TCC

\titulo{Projeto de Graduação} %TCC

\renewcommand{\thesection}{\arabic{chapter}.\arabic{section}}

\begin{document}



% ----------------------------------------------------------
% ELEMENTOS PRÉ-TEXTUAIS
% ----------------------------------------------------------
%\pretextual
\imprimircapa
\imprimirfolhaderosto

%\begin{resumo} %TCC:

 %\vspace{\onelineskip}

 %\noindent
 %\textbf{Palavras-chave}: Acordo de Nível de Serviço (SLA), Monitoramento, Multicaminhos,
 %OpenFlow, Qualidade de Serviço (QoS), Redes Definidas por Software. %TCC:
%\end{resumo}

%% ---
%% inserir lista de símbolos, se for adequado ao trabalho. %TCC:
%% ---


%\begin{siglas}
%\end{siglas}

%\begin{simbolos}
%  \item[$ \Gamma $] Letra grega Gama
%  \item[$ \Lambda $] Lambda
%  \item[$ \zeta $] Letra grega minúscula zeta
%  \item[$ \in $] Pertence
%\end{simbolos}
%% ---

% ---
% inserir o sumario
% ---
\pdfbookmark[0]{\contentsname}{toc}
\tableofcontents*
\cleardoublepage
% ---


% ----------------------------------------------------------
% ELEMENTOS TEXTUAIS
% ----------------------------------------------------------
\textual

% ######################### INICIO ###########################################
\chapter{Introdução}

Com o avanço da tecnologia a nossa interação com a computação se tornou diária, mesmo
nos mínimos detalhes sempre acabamos interagindo com algum software no dia a dia. Com
esse grande aumento na utilização de softwares e do aprimoramento de novas tecnologias, 
surge a necessidade de aplicações resilientes e tolerantes a falhas. 

Uma aplicação tolerante a falhas consegue manter o funcionamento mesmo com a eventual 
falha de um de seus componentes. Como é possível existirem eventos inesperados que podem
causar a eventual falha, - ou \textit{crash} - de um servidor, são necessários métodos
para garantir a consistência quando a aplicação retornar com seu funcionamento
normal. 

Uma maneira de possuir uma aplicação tolerante a falhas é a replicação de máquinas
de estado, onde o serviço é modelado como uma máquina de estados determinísticos, e o
serviço é executado em cada réplica \citep{santos2012state}, dessa maneira é mantida
a consistência da aplicação. No centro da replicação de máquina de estados, existe o 
consenso, onde o conjunto de réplicas deve chegar em um consenso e decidir sobre 
qual o estado válido, decidir sobre qual transição será realizada. 

Para obtenção do consenso e da criação de uma aplicação tolerante a falhas, o algoritmo
Paxos é extensivamente utilizado com esse intuito. O algoritmo realiza uma custosa 
trocas de mensagens entre seus agentes. Por se tratar de um ambiente distribuído 
as mensagens podem demorar um tempo arbitrário para chegar ao destinatário ou se 
perder durante a transmissão. Para se conseguir uma melhora na performance pode-se 
realizar a implementação do Paxos no dispositivos físicos da rede
\citep{dang2016paxos}.

Com a utilização da linguagem P4, é aberta a possibilidade de se programar dispositivos
de rede. Com a utilização de uma tabela \textit{match+action} é possível 
realizar um processamento customizado dos pacotes sendo transportados.

\section{Objetivos}
O presente trabalho visa realizar a implementação do algoritmo Paxos nos dispositivos
de rede, especificamente \textit{switches}, utilizando a linguagem $P4_{16}$. 
Desta maneira, conseguindo uma melhora na performance do algoritmo, pois uma parte 
que seria executado em um servidor se moveria para ser executado nesses dispositivos.

Existe um trabalho na área, publicado em 2016 \citep{dang2016paxos}. Neste trabalho,
os pesquisadores utilizaram $P4_{14}$ na versão 1.0.2, e deixaram pontos que podem ser
melhorados no trabalho com a evolução da linguagem, que atualmente se encontra na versão
$P4_{16}$ 1.0.0, que é uma revisão da P4 de 2016 \citep{paxos16spec}.

\subsection{Objetivos Gerais}
Os objetivos iniciais do projeto, seriam, de modo claro, replicar o trabalho 
desenvolvido anteriormente e implementar os pontos de melhora citados pelos autores.

Com uma gama maior de opções com a evolução linguagem $P4_{16}$, 
analisar pontos passíveis de melhora no presente projeto e realizar uma 
análise e implementação de tais.

\section{Justificativas}
\section{Método}

%######################### FUNDAMENTACAO ##################################
\chapter{Fundamentação Teórica}
Esse capítulo manterá o foco em detalhar e explicar as técnicas utilizadas
para realização do projeto. Não serão abordadas questões de implementação,
serão explicadas somente de modo teórico, afim de auxiliar o entendimento do leitor.

\section{Referencial Teórico}
Para seguirmos uma linha de raciocínio linear, e melhorar entendimento da teoria
para compreender o projeto, será abordado inicialmente informações sobre o 
algoritmo Paxos, seu modo de operar e flexibilidade para resolução de problemas.

Em seguida será abordado como funcionam linguagens de programação para o plano de
dados em dispositivos de redes sendo realizado um paralelo com a linguagem P4, 
especificamente a versão utilizada para o desenvolvimento deste projeto e algumas
outras opções disponíveis.

E para finalizar o referencial teórico, será explicado sobre o Mininet, de que
maneira funciona e qual o intuito na sua utilização no projeto.

\subsection{Paxos}
\subsection{P4}
\subsection{Mininet}


\section{Trabalhos Correlatos}


%######################### DESENVOLVIMENTO ##################################
\chapter{Desenvolvimento}
 
\subsection{Arquitetura do projeto}
\subsection{Arquitetura da rede}
\subsubsection{Cabeçalhos}
\subsubsection{Experimento}
\subsubsection{Código}
\subsubsection{Testes}

%######################### CONCLUSAO ##################################
\section{Conclusão}



\bibliography{references.bib}
\bibliographystyle{plain}

\end{document}

