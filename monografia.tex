\documentclass[12pt,
openright, 
oneside,
%twoside, %TCC: Se seu texto tem mais de 100 páginas, descomente esta linha e comente a anterior
a4paper,
brazil]{facom-ufu-abntex2}

%\usepackage[disable]{todonotes}
\usepackage[colorinlistoftodos]{todonotes}	%use a linha anterior para esconder os todos.
\usepackage{listings}
\usepackage{forest}
\usepackage{listings}

\newcommand{\codigo}{
\lstset{
	language=C++,
    basicstyle=\ttfamily\tiny,
    stringstyle=\color{red},
    commentstyle=\color{green},
    morecomment=[s][\color{blue}]{/**}{*/},
    extendedchars=true,
    showspaces=false,
    showstringspaces=false,
    numbers=left,
    numberstyle=\tiny,
    breaklines=true,
    breakautoindent=true,
    breakatwhitespace=true,
}}

\renewcommand{\lstlistingname}{Listagem}

\autor{José Augusto Bolina Lucas} %TCC
\data{2018}
\orientador{Lásaro Jonas Camargos} %TCC

\titulo{Projeto de Graduação} %TCC

\renewcommand{\thesection}{\arabic{chapter}.\arabic{section}}

\begin{document}



% ----------------------------------------------------------
% ELEMENTOS PRÉ-TEXTUAIS
% ----------------------------------------------------------
%\pretextual
\imprimircapa
\imprimirfolhaderosto

%\begin{resumo} %TCC:

 %\vspace{\onelineskip}

 %\noindent
 %\textbf{Palavras-chave}: Acordo de Nível de Serviço (SLA), Monitoramento, Multicaminhos,
 %OpenFlow, Qualidade de Serviço (QoS), Redes Definidas por Software. %TCC:
%\end{resumo}

%% ---
%% inserir lista de símbolos, se for adequado ao trabalho. %TCC:
%% ---


%\begin{siglas}
%\end{siglas}

%\begin{simbolos}
%  \item[$ \Gamma $] Letra grega Gama
%  \item[$ \Lambda $] Lambda
%  \item[$ \zeta $] Letra grega minúscula zeta
%  \item[$ \in $] Pertence
%\end{simbolos}
%% ---

% ---
% inserir o sumario
% ---
\pdfbookmark[0]{\contentsname}{toc}
\tableofcontents*
\cleardoublepage
% ---


% ----------------------------------------------------------
% ELEMENTOS TEXTUAIS
% ----------------------------------------------------------
\textual

% ######################### INICIO ###########################################
\chapter{Introdução}

Com o avanço da tecnologia a nossa interação com a computação se tornou diária, mesmo
nos mínimos detalhes sempre acabamos interagindo com algum softwares no dia a dia. Do
grande aumento na utilização de softwares e do aprimoramento de novas tecnologias, surge
a necessidade de aplicações tolerantes a falhas. 

Uma aplicação tolerante a falhas consegue manter o funcionamento mesmo com a eventual 
falha de um de seus componente, e assim, esperar que esse componente retorne para 
funcionamento em algum momento. Como é possível existirem eventos inesperados que podem
causar a eventual falha, ou \textit{crash}, de um servidor, com o seu retorno para 
funcionamento é necessária uma intensa comunicação para que a aplicação continue
consistente com os eventos que ocorrem durante o período de falha.

Uma maneira de possuir uma aplicação tolerante a falhas é a replicação de máquinas
de estado, onde o serviço é modelo como uma máquina de estados determinística, e o
serviço é executado em cada réplica \citep{santos2012state}, dessa maneira é mantida
a consistência da aplicação. No centro da replicação de máquina de estados existe o 
consenso, onde o conjunto de réplicas deve chegar em um consenso sobre qual o estado 
válido, qual transição será realizada. 

Para obtenção do consenso e da criação de uma aplicação tolerante a falhas, o algoritmo
Paxos é extensivamente com esse intuito. O algoritmo realiza uma extensa e custosa 
trocas de mensagens entre seus agentes, com as mensagens podendo demorar um tempo 
arbitrário para chegar ao destinatário ou se perder. Para se conseguir uma melhora na
performance pode-se realizar a implementação do Paxos no dispositivos fisícos da rede
\citep{dang2016paxos}.

Com a utilização da linguagem P4 é possível programar o plano de dados nos 
\textit{switches} da rede, onde é possível customizar o processamentos do pacotes
nos dispositivos. 


\section{Objetivos}
O presente trabalho visa realizar a implementação do algoritmo Paxos nos dispositivos
de rede, especificamente \textit{switches} utilizando a linguagem $P4_{16}$. 
Deste maneira, conseguindo uma melhora na performance do algoritmo, pois uma parte 
do algoritmo se moveria de um servidor para ser executado nesses dispositivos.

Dispositivos como o \textit{switch} não possuem a capacidade de criar novas mensagens,
somente encaminhar, assim, as mensagens que seriam trocadas entre os agentes e precisariam
de uma lógica para processamento nos servidores passariam as ser processadas pelo 
\textit{switch}, "no fio" \citep{dang2016paxos}.

Existe um trabalho na área, publicado em 2016 \citep{dang2016paxos}. Neste trabalho,
os pesquisadores utilizaram $P4_{14}$ na versão 1.0.2, e deixaram pontos que podem ser
melhorados no trabalho com a evolução da linguagem, que atualmente se encontra na versão
$P4_{16}$ 1.0.0 que é uma revisão da P4 de 2016 \citep{paxos16spec}.

\subsection{Objetivos Gerais}
Os objetivos iniciais do projeto, seriam, de modo geral, replicar o trabalho 
desenvolvido anteriormente e implementar os pontos passíveis de melhora citados
pelos autores.

Com uma gama maior de opções com a linguagem $P4_{16}$ analisar pontos passíveis
de melhora no presente projeto e realizar a implementação de tais.

\section{Justificativas}
\section{Método}

%######################### FUNDAMENTACAO ##################################
\chapter{Fundamentação Teórica}
Esse capítulo manterá um foco em detalhar e explicar as técnicas utilizadas
para realização do projeto. Não serão abordadas questões de implementação,
serão explicadas somente de modo teórico para entendimento do leitor.

\section{Referencial Teórico}
Para seguirmos uma linha de raciocínio linear e melhor entendimento da teoria
para compreender o projeto, será abordado inicialmente informações sobre o 
algoritmo Paxos, seu modo de operar e flexibilidade para resolução de problemas.

Em seguida será abordado como funcionam linguagem de programação para o plano de
dados em dispositivos de redes sendo realizado um paralelo com a linguagem P4, 
especificamente a versão utilizada para o desenvolvimento deste projeto.

E para finalizar o referencial teórico, será explicado sobre o Mininet, de que
maneira funciona e qual o intuito na sua utilização no projeto.

\subsection{Paxos}
\subsection{P4}
\subsection{Mininet}


\section{Trabalhos Correlatos}


%######################### DESENVOLVIMENTO ##################################
\chapter{Desenvolvimento}
Nesse capítulo veremos o programa desenvolvido utilizando p4, entre outras tecnologias, e explicaremos sobre o ambiente de simulação, topologia entre outros assuntos.
 
\subsection{Arquitetura do projeto}
\subsection{Arquitetura da rede}
\subsubsection{Cabeçalhos}
\subsubsection{Experimento}
\subsubsection{Código}
\subsubsection{Testes}

%######################### CONCLUSAO ##################################
\section{Conclusão}



\bibliography{references.bib}
\bibliographystyle{plain}

\end{document}

