\documentclass[12pt,
openright, 
oneside,
%twoside, %TCC: Se seu texto tem mais de 100 páginas, descomente esta linha e comente a anterior
a4paper,
brazil]{facom-ufu-abntex2}

%\usepackage[disable]{todonotes}
\usepackage[colorinlistoftodos]{todonotes}	%use a linha anterior para esconder os todos.
\usepackage{listings}
\usepackage{forest}
\usepackage{listings}

\newcommand{\codigo}{
\lstset{
	language=C++,
    basicstyle=\ttfamily\tiny,
    stringstyle=\color{red},
    commentstyle=\color{green},
    morecomment=[s][\color{blue}]{/**}{*/},
    extendedchars=true,
    showspaces=false,
    showstringspaces=false,
    numbers=left,
    numberstyle=\tiny,
    breaklines=true,
    breakautoindent=true,
    breakatwhitespace=true,
}}

\renewcommand{\lstlistingname}{Listagem}

\autor{José Augusto Bolina Lucas} %TCC
\data{2018}
\orientador{Lásaro Jonas Camargos} %TCC

\titulo{Projeto de Graduação} %TCC

\renewcommand{\thesection}{\arabic{chapter}.\arabic{section}}

\begin{document}



% ----------------------------------------------------------
% ELEMENTOS PRÉ-TEXTUAIS
% ----------------------------------------------------------
%\pretextual
\imprimircapa
\imprimirfolhaderosto

%\begin{resumo} %TCC:

 %\vspace{\onelineskip}

 %\noindent
 %\textbf{Palavras-chave}: Acordo de Nível de Serviço (SLA), Monitoramento, Multicaminhos,
 %OpenFlow, Qualidade de Serviço (QoS), Redes Definidas por Software. %TCC:
%\end{resumo}

%% ---
%% inserir lista de símbolos, se for adequado ao trabalho. %TCC:
%% ---


\begin{siglas}
\end{siglas}

%\begin{simbolos}
%  \item[$ \Gamma $] Letra grega Gama
%  \item[$ \Lambda $] Lambda
%  \item[$ \zeta $] Letra grega minúscula zeta
%  \item[$ \in $] Pertence
%\end{simbolos}
%% ---

% ---
% inserir o sumario
% ---
\pdfbookmark[0]{\contentsname}{toc}
\tableofcontents*
\cleardoublepage
% ---


% ----------------------------------------------------------
% ELEMENTOS TEXTUAIS
% ----------------------------------------------------------
\textual

% ----------------------------------------------------------
% Introdução
% ----------------------------------------------------------
\chapter{Introdução}
Com o avanço da tecnologia a nossa interação com a computação se tornou diária, mesmo
nos mínimos detalhes sempre acabamos interagindo com algum softwares no dia a dia. Do
grande aumento na utilização de softwares e do aprimoramento de novas tecnologias, surge
a necessidade de aplicações tolerantes a falhas. \\

Uma aplicação tolerante a falhas consegue manter o funcionamento mesmo com a eventual 
falha de um de seus componente, e assim, esperar que esse componente retorne para 
funcionamento em algum momento. Como é possível existirem eventos inesperados que podem
causar a eventual falha, ou \textit{crash}, de um servidor, com o seu retorno para 
funcionamento é necessária uma intensa comunicação para que a aplicação continue
consistente com os eventos que ocorrem durante o período de falha.\\

Uma maneira de possuir uma aplicação tolerante a falhas é a replicação de máquinas
de estado, 

\section{Objetivos}
\subsection{Objetivos Gerais}
\subsection{Objetivos Específicos}

\section{Justificativas}
\section{Método}

%######################### FUNDAMENTACAO ##################################
\chapter{Fundamentação Teórica}
\section{Referencial Teórico}
Aqui veremos sobre as teorias e tecnologias utilizadas no trabalho.
\subsection{Paxos}
\subsection{P4}
\subsection{Mininet}


\section{Trabalhos Correlatos}


%######################### DESENVOLVIMENTO ##################################
\chapter{Desenvolvimento}
Nesse capítulo veremos o programa desenvolvido utilizando p4, entre outras tecnologias, e explicaremos sobre o ambiente de simulação, topologia entre outros assuntos.
 
\subsection{Arquitetura do projeto}
\subsection{Arquitetura da rede}
\subsubsection{Cabeçalhos}
\subsubsection{Experimento}
\subsubsection{Código}
\subsubsection{Testes}

%######################### CONCLUSAO ##################################
\section{Conclusão}
\end{document}

